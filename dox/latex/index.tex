\section*{S\-N\-A\-P\-: S\-N (Discrete Ordinates) Application Proxy}

\subsection*{Description}

S\-N\-A\-P serves as a proxy application to model the performance of a modern discrete ordinates neutral particle transport application. S\-N\-A\-P may be considered an update to \href{http://www.ccs3.lanl.gov/PAL/software.shtml}{\tt Sweep3\-D}, intended for hybrid computing architectures. It is modeled off the Los Alamos National Laboratory code P\-A\-R\-T\-I\-S\-N. P\-A\-R\-T\-I\-S\-N solves the linear Boltzmann transport equation (T\-E), a governing equation for determining the number of neutral particles (e.\-g., neutrons and gamma rays) in a multi-\/dimensional phase space. S\-N\-A\-P itself is not a particle transport application; S\-N\-A\-P incorporates no actual physics in its available data, nor does it use numerical operators specifically designed for particle transport. Rather, S\-N\-A\-P mimics the computational workload, memory requirements, and communication patterns of P\-A\-R\-T\-I\-S\-N. The equation it solves has been composed to use the same number of operations, use the same data layout, and load elements of the arrays in approximately the same order. Although the equation S\-N\-A\-P solves looks similar to the T\-E, it has no real world relevance.

The solution to the time-\/dependent T\-E is a \char`\"{}flux\char`\"{} function of seven independent variables\-: three spatial (3-\/\-D spatial mesh), two angular (set of discrete ordinates, directions in which particles travel), one energy (particle speeds binned into \char`\"{}groups\char`\"{}), and one temporal. P\-A\-R\-T\-I\-S\-N, and therefore S\-N\-A\-P, uses domain decomposition over these dimensions to coherently distribute the data and the tasks associated with solving the equation. The parallelization strategy is expected to be the most efficient compromise between computing resources and the iterative strategy necessary to converge the flux.

The iterative strategy is comprised of a set of two nested loops. These nested loops are performed for each step of a time-\/dependent calculation, wherein any particular time step requires information from the preceding one. No parallelization is performed over the temporal domain. However, for time-\/dependent calculations two copies of the unknown flux must be stored, each copy an array of the six remaining dimensions. The outer iterative loop involves solving for the flux over the energy domain with updated information about coupling among the energy groups. Typical calculations require tens to hundreds of groups, making the energy domain suitable for threading with the node's (or nodes') provided accelerator. The inner loop involves sweeping across the entire spatial mesh along each discrete direction of the angular domain. The spatial mesh may be immensely large. Therefore, S\-N\-A\-P spatially decomposes the problem across nodes and communicates needed information according to the K\-B\-A method. K\-B\-A is a transport-\/specific application of general parallel wavefront methods. Lastly, although K\-B\-A efficiency is improved by pipelining operations according to the angle, current chipsets operate best with vectorized operations. During a mesh sweep, S\-N\-A\-P operations are vectorized over angles to take advantage of the modern hardware.

S\-N\-A\-P should be tested with problem sizes that accurately reflect the types of calculations P\-A\-R\-T\-I\-S\-N frequently handles. The spatial domain shall be decomposed to 2,000--4,000 cells per node (M\-P\-I rank). Each node will own all the energy groups and angles for that group of cells; typical calculations feature 10--100 energy groups and as few as 100 to as many as 2,000 angles. Moreover, sufficient memory must be provided to store two full copies of the solution vector for time-\/dependent calculations. The preceding parameters assume current trends in available per core memory. Significant advances or detriments affecting this assumption shall require reconsideration of appropriate parameters per compute node.

\subsection*{Compilation}

S\-N\-A\-P has been written to the Fortran 90/95 standard. It has been successfully built with, but not necessarily limited to, gfortran and ifort. Moreover, the code has been built with the profiling tool \href{https://github.com/losalamos/byfl}{\tt Byfl}. The accompanying Makefile retains some of the old make options for different build types. However, the current build system depends on the availability of M\-P\-I and Open\-M\-P libraries. Builds without these libraries will require modification to the source code to remove related subroutine calls and directives.

M\-P\-I implementations typically suggest using a \char`\"{}wrapper\char`\"{} compiler to compile the code. S\-N\-A\-P has been built and tested with Open\-M\-P\-I. Open\-M\-P\-I allows one to set the underlying Fortran compiler with the environment variable O\-M\-P\-I\-\_\-\-F\-C, where the variable is set to the (path and) compiler of choice, e.\-g., ifort, gfortran, etc.

The makefile currently uses\-: \begin{DoxyVerb}FORTRAN = mpif90
\end{DoxyVerb}


and all testing has been performed with \begin{DoxyVerb}OMPI_FC = [path]/ifort
\end{DoxyVerb}


Fortran compilation flags can be set according to the underlying compiler. The current flags are set for the ifort compiler and using Open\-M\-P for parallel threading. \begin{DoxyVerb}TARGET = snap
FFLAGS = -03 -openmp
FFLAG2 =
\end{DoxyVerb}


where {\ttfamily F\-F\-L\-A\-G2} is reserved for additional flags that may need applied differently, depending on the compiler. To make S\-N\-A\-P with these default settings, simply type \begin{DoxyVerb}make
\end{DoxyVerb}


on the command line within the S\-N\-A\-P directory.

A debugging version of S\-N\-A\-P can be built by typing \begin{DoxyVerb}make OPT=no
\end{DoxyVerb}


on the command line. The unoptimized, debugging version of S\-N\-A\-P features bounds checking, back-\/tracing an error, and the necessary debug compiler flags. With ifort, these flags appear as\-: \begin{DoxyVerb}FFLAGS = -g -O0 -check bounds -traceback -openmp
FFLAG2 =
\end{DoxyVerb}


The values for these compilation variables have been modified for various Fortran compilers and the Makefile provides details of what has been used previously. These lines are commented out for clarity at this time and to ensure that changes to the build system are made carefully before attempting to rebuild with a different compiler.

The S\-N\-A\-P directory can be cleaned up of its module and object files if the user desires with\-: \begin{DoxyVerb}make clean
\end{DoxyVerb}


This removes all the {\ttfamily $\ast$.mod} and {\ttfamily $\ast$.o} files, as well as {\ttfamily $\ast$.bc} files from Byfl builds. Moreover, it will enforce complete recompilation of all files upon the next instance of {\ttfamily make} or {\ttfamily make O\-P\-T=no}. Currently, there is no separate directory for the compilation files of separate optimized and unoptimized builds. The user must do a {\ttfamily make clean} before building the code if the previous build used the opposite command.

Lastly, a line count report is generated with\-: \begin{DoxyVerb}make count
\end{DoxyVerb}


The line count report excludes blank lines and comments. It counts the number of code lines in all {\ttfamily $\ast$.f90} files and sums the results. The information is printed to the the {\ttfamily Lines} file.

\subsection*{Usage}

Because S\-N\-A\-P currently requires building with M\-P\-I, to execute S\-N\-A\-P, use the following command\-: \begin{DoxyVerb}mpirun -np [#] [path]/snap [infile] [outfile]
\end{DoxyVerb}


This command will automatically run with the number of threads specified by the input file, which is used to set the number of Open\-M\-P threads, overwriting any environment variable to set the number of threads. Testing has shown that to ensure proper concurrency of work, the above command can be modified to \begin{DoxyVerb}mpirun -cpus-per-proc [#threads] -np [#procs] [path]/snap [infile] [outfile]
\end{DoxyVerb}


Lastly, a user may wish to test the various thread affinity settings used to bind threads to processing elements. Testing has been done with a disabled Intel thread affinity interface. \begin{DoxyVerb}setenv KMP_AFFINITY disabled (csh)
\end{DoxyVerb}


The command line is read for the input/output file names. If one of the names is missing, the code will not execute. Moreover, the output file overwrites any pre-\/existing files of the same name.

\subsection*{Sample Input}

The following is a sample input of a S\-N\-A\-P job. Several other examples are provided as part of the small set of regression testing. For more information about the valid range of values and descriptions of the input variables, please see the user manual. \begin{DoxyVerb}! Input from namelist
&invar
  nthreads=2
  npey=2
  npez=2
  ndimen=3
  nx=4
  lx=1.0
  ny=4
  ly=1.0
  nz=4
  lz=1.0
  ichunk=4
  nmom=2
  nang=8
  ng=2
  mat_opt=0
  src_opt=0
  timedep=0
  it_det=0
  tf=0.0
  nsteps=1
  iitm=5
  oitm=100
  epsi=1.E-4
  fluxp=0
  scatp=0
  fixup=0
/
\end{DoxyVerb}


\subsection*{Sample Output}

The following is the corresponding output to the above case. A brief outline of the output file contents is version and run time information, echo of input (or default) values of the namelist variables, echo of relevant parameters after setup, iteration monitor, mid-\/plane flux output, and the timing summary. Warning and error messages may be printed throughout the output file to alert the user to some problem with the execution. Unlike errors, warnings do not cause program termination. \begin{DoxyVerb} SNAP: SN (Discrete Ordinates) Application Proxy
 Version Number..  1.00 
 Version Date..  03-07-2013
 Ran on  3-13-2013 at time 14: 5:39

********************************************************************************

          Input Echo - Values from input or default
********************************************************************************

  NML=invar
     npey=     2
     npez=     2
     ichunk=     4
     nthreads=     2
     ndimen=  3
     nx=     4
     ny=     4
     nz=     4
     lx=  1.0000E+00
     ly=  1.0000E+00
     lz=  1.0000E+00
     nmom=   2
     nang=    8
     ng=    2
     mat_opt=  0
     src_opt=  0
     scatp=  0
     epsi=  1.0000E-04
     iitm=   5
     oitm=  100
     timedep=  0
     tf=  0.0000E+00
     nsteps=     1
     it_det=  0
     fluxp=  0
     fixup=  0

********************************************************************************

          Calculation Run-time Parameters Echo
********************************************************************************

  Geometry
    ndimen = 3
    nx =     4
    ny =     4
    nz =     4
    lx =  1.0000E+00
    ly =  1.0000E+00
    lz =  1.0000E+00
    dx =  2.5000E-01
    dy =  2.5000E-01
    dz =  2.5000E-01

  Sn
    nmom = 2
    nang =    8
    noct = 8

    w =  1.5625E-02   ... uniform weights

          mu              eta               xi
     6.25000000E-02   9.37500000E-01   3.42326598E-01
     1.87500000E-01   8.12500000E-01   5.51985054E-01
     3.12500000E-01   6.87500000E-01   6.55505530E-01
     4.37500000E-01   5.62500000E-01   7.01560760E-01
     5.62500000E-01   4.37500000E-01   7.01560760E-01
     6.87500000E-01   3.12500000E-01   6.55505530E-01
     8.12500000E-01   1.87500000E-01   5.51985054E-01
     9.37500000E-01   6.25000000E-02   3.42326598E-01

  Material Map
    mat_opt = 0   &ndash;>   nmat = 1
    Base material (default for every cell) = 1

  Source Map
    src_opt = 0
    Source strength per cell (where applied) = 1.0
    Source map:
        Starting cell: (     1,     1,     1 )
        Ending cell:   (     4,     4,     4 )

  Pseudo Cross Sections Data
    ng =   2

    Material 1
    Group         Total         Absorption      Scattering
       1       1.000000E+00    5.000000E-01    5.000000E-01
       2       1.010000E+00    5.050000E-01    5.050000E-01

  Solution Control Parameters
    epsi =  1.0000E-04
    iitm =   5
    oitm =  100
    timedep = 0
    it_det = 0
    fluxp = 0
    fixup = 0

  Parallelization Parameters
    npey =     2
    npez =     2
    nthreads =    2

          Thread Support Level
           0 - MPI_THREAD_SINGLE
           1 - MPI_THREAD_FUNNELED
           2 - MPI_THREAD_SERIALIZED
           3 - MPI_THREAD_MULTIPLE
    thread_level =  0

********************************************************************************

          Iteration Monitor
********************************************************************************
  Outer
    1    Dfmxo= 5.5956E-01    No. Inners=   10
    2    Dfmxo= 1.3849E-01    No. Inners=    8
    3    Dfmxo= 2.8164E-03    No. Inners=    6

  No. Outers=   3    No. Inners=   24

********************************************************************************

          Calculation Final Scalar Flux Solution
********************************************************************************

 ***********************************
  Group=   1   Z Mid-Plane=    3
 ***********************************

     y    x    1      x    2      x    3      x    4
     4  3.1216E-01  3.9666E-01  3.9666E-01  3.1216E-01
     3  3.9683E-01  5.0638E-01  5.0638E-01  3.9683E-01
     2  3.9683E-01  5.0638E-01  5.0638E-01  3.9683E-01
     1  3.1216E-01  3.9666E-01  3.9666E-01  3.1216E-01

********************************************************************************

 ***********************************
  Group=   2   Z Mid-Plane=    3
 ***********************************

     y    x    1      x    2      x    3      x    4
     4  3.8033E-01  4.9130E-01  4.9130E-01  3.8033E-01
     3  4.9210E-01  6.3838E-01  6.3838E-01  4.9210E-01
     2  4.9210E-01  6.3838E-01  6.3838E-01  4.9210E-01
     1  3.8033E-01  4.9130E-01  4.9130E-01  3.8033E-01

********************************************************************************

          Timing Summary
********************************************************************************

  Code Section                          Time (seconds)
 **************                        ****************
    Parallel Setup                       1.0433E-03
    Input                                4.7398E-04
    Setup                                5.8007E-04
    Solve                                2.2089E-03
       Parameter Setup                   2.1720E-04
       Outer Source                      1.8597E-05
       Inner Iterations                  1.8098E-03
          Inner Source                   3.0279E-05
          Transport Sweeps               1.4706E-03
          Inner Misc Ops                 3.0899E-04
       Solution Misc Ops                 1.6332E-04
    Output                               2.6178E-04
  Total Execution time                   1.5435E-02

  Grind Time (nanoseconds)         2.2471E+01

********************************************************************************
\end{DoxyVerb}


Additional outputs in the form of {\ttfamily slgg} and {\ttfamily flux} files are available when requested according to the {\ttfamily scatp} and {\ttfamily fluxp} input variables, respectively.

\subsection*{License}

Los Alamos National Security, L\-L\-C owns the copyright to \char`\"{}\-S\-N\-A\-P\-: S\-N (\-Discrete Ordinates) Application Proxy, Version 1.\-x (\-C13087)\char`\"{}. The license is B\-S\-D with standard clauses regarding indicating modifications before future redistribution\-:

Copyright (c) 2013, Los Alamos National Security, L\-L\-C All rights reserved.

Copyright 2013. Los Alamos National Security, L\-L\-C. This software was produced under U.\-S. Government contract D\-E-\/\-A\-C52-\/06\-N\-A25396 for Los Alamos National Laboratory (L\-A\-N\-L), which is operated by Los Alamos National Security, L\-L\-C for the U.\-S. Department of Energy. The U.\-S. Government has rights to use, reproduce, and distribute this software. N\-E\-I\-T\-H\-E\-R T\-H\-E G\-O\-V\-E\-R\-N\-M\-E\-N\-T N\-O\-R L\-O\-S A\-L\-A\-M\-O\-S N\-A\-T\-I\-O\-N\-A\-L S\-E\-C\-U\-R\-I\-T\-Y, L\-L\-C M\-A\-K\-E\-S A\-N\-Y W\-A\-R\-R\-A\-N\-T\-Y, E\-X\-P\-R\-E\-S\-S O\-R I\-M\-P\-L\-I\-E\-D, O\-R A\-S\-S\-U\-M\-E\-S A\-N\-Y L\-I\-A\-B\-I\-L\-I\-T\-Y F\-O\-R T\-H\-E U\-S\-E O\-F T\-H\-I\-S S\-O\-F\-T\-W\-A\-R\-E. If software is modified to produce derivative works, such modified software should be clearly marked, so as not to confuse it with the version available from L\-A\-N\-L.

Additionally, redistribution and use in source and binary forms, with or without modification, are permitted provided that the following conditions are met\-:


\begin{DoxyItemize}
\item Redistributions of source code must retain the above copyright notice, this list of conditions and the following disclaimer.
\item Redistributions in binary form must reproduce the above copyright notice, this list of conditions and the following disclaimer in the documentation and/or other materials provided with the distribution.
\item Neither the name of Los Alamos National Security, L\-L\-C, Los Alamos National Laboratory, L\-A\-N\-L, the U.\-S. Government, nor the names of its contributors may be used to endorse or promote products derived from this software without specific prior written permission.
\end{DoxyItemize}

T\-H\-I\-S S\-O\-F\-T\-W\-A\-R\-E I\-S P\-R\-O\-V\-I\-D\-E\-D B\-Y L\-O\-S A\-L\-A\-M\-O\-S N\-A\-T\-I\-O\-N\-A\-L S\-E\-C\-U\-R\-I\-T\-Y, L\-L\-C A\-N\-D C\-O\-N\-T\-R\-I\-B\-U\-T\-O\-R\-S \char`\"{}\-A\-S I\-S\char`\"{} A\-N\-D A\-N\-Y E\-X\-P\-R\-E\-S\-S O\-R I\-M\-P\-L\-I\-E\-D W\-A\-R\-R\-A\-N\-T\-I\-E\-S, I\-N\-C\-L\-U\-D\-I\-N\-G, B\-U\-T N\-O\-T L\-I\-M\-I\-T\-E\-D T\-O, T\-H\-E I\-M\-P\-L\-I\-E\-D W\-A\-R\-R\-A\-N\-T\-I\-E\-S O\-F M\-E\-R\-C\-H\-A\-N\-T\-A\-B\-I\-L\-I\-T\-Y A\-N\-D F\-I\-T\-N\-E\-S\-S F\-O\-R A P\-A\-R\-T\-I\-C\-U\-L\-A\-R P\-U\-R\-P\-O\-S\-E A\-R\-E D\-I\-S\-C\-L\-A\-I\-M\-E\-D. I\-N N\-O E\-V\-E\-N\-T S\-H\-A\-L\-L L\-O\-S A\-L\-A\-M\-O\-S N\-A\-T\-I\-O\-N\-A\-L S\-E\-C\-U\-R\-I\-T\-Y, L\-L\-C O\-R C\-O\-N\-T\-R\-I\-B\-U\-T\-O\-R\-S B\-E L\-I\-A\-B\-L\-E F\-O\-R A\-N\-Y D\-I\-R\-E\-C\-T, I\-N\-D\-I\-R\-E\-C\-T, I\-N\-C\-I\-D\-E\-N\-T\-A\-L, S\-P\-E\-C\-I\-A\-L, E\-X\-E\-M\-P\-L\-A\-R\-Y, O\-R C\-O\-N\-S\-E\-Q\-U\-E\-N\-T\-I\-A\-L D\-A\-M\-A\-G\-E\-S (I\-N\-C\-L\-U\-D\-I\-N\-G, B\-U\-T N\-O\-T L\-I\-M\-I\-T\-E\-D T\-O, P\-R\-O\-C\-U\-R\-E\-M\-E\-N\-T O\-F S\-U\-B\-S\-T\-I\-T\-U\-T\-E G\-O\-O\-D\-S O\-R S\-E\-R\-V\-I\-C\-E\-S; L\-O\-S\-S O\-F U\-S\-E, D\-A\-T\-A, O\-R P\-R\-O\-F\-I\-T\-S; O\-R B\-U\-S\-I\-N\-E\-S\-S I\-N\-T\-E\-R\-R\-U\-P\-T\-I\-O\-N) H\-O\-W\-E\-V\-E\-R C\-A\-U\-S\-E\-D A\-N\-D O\-N A\-N\-Y T\-H\-E\-O\-R\-Y O\-F L\-I\-A\-B\-I\-L\-I\-T\-Y, W\-H\-E\-T\-H\-E\-R I\-N C\-O\-N\-T\-R\-A\-C\-T, S\-T\-R\-I\-C\-T L\-I\-A\-B\-I\-L\-I\-T\-Y, O\-R T\-O\-R\-T (I\-N\-C\-L\-U\-D\-I\-N\-G N\-E\-G\-L\-I\-G\-E\-N\-C\-E O\-R O\-T\-H\-E\-R\-W\-I\-S\-E) A\-R\-I\-S\-I\-N\-G I\-N A\-N\-Y W\-A\-Y O\-U\-T O\-F T\-H\-E U\-S\-E O\-F T\-H\-I\-S S\-O\-F\-T\-W\-A\-R\-E, E\-V\-E\-N I\-F A\-D\-V\-I\-S\-E\-D O\-F T\-H\-E P\-O\-S\-S\-I\-B\-I\-L\-I\-T\-Y O\-F S\-U\-C\-H D\-A\-M\-A\-G\-E.

\subsection*{Classification}

S\-N\-A\-P is Unclassified and contains no Unclassified Controlled Nuclear Information. It has been assigned Los Alamos Computer Code number L\-A-\/\-C\-C-\/13-\/016.

\subsection*{Authors}


\begin{DoxyItemize}
\item Joe Zerr, C\-C\-S-\/2, Los Alamos National Laboratory
\item Randal Baker, C\-C\-S-\/2, Los Alamos National Laboratory
\end{DoxyItemize}

Questions and issues\-: \href{mailto:snap@lanl.gov}{\tt snap@lanl.\-gov}

$\ast$/ 